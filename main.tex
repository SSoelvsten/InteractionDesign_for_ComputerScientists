%%%%%%%%%%%%%%%%%%%%%%%%%%%
% Set up document         %
%%%%%%%%%%%%%%%%%%%%%%%%%%%
\documentclass[a4, english, twoside]{article}

% Import SSoelvsten's LaTeX preamble
% github.com/SSoelvsten/LaTeX-Preamble_and_Examples
\usepackage{import}
\subimport{../LaTeX-Preamble_and_Examples/preamble/}{preamble_en.tex}

% New settings for this specific document

% Use these for 'definitions' in the theory section
\newtheorem{concept}[theorem]{Concept}
\newtheorem{framework}[theorem]{Framework}
\newtheorem{method}[theorem]{Method}

% Indexing
% To index something, write \index{keyword}
\usepackage{makeidx}
\usepackage[columns=2]{idxlayout}
\makeindex

%%%%%%%%%%%%%%%%%%%%%%%%%
% Document starts here! %
%%%%%%%%%%%%%%%%%%%%%%%%%

\begin{document}
\settitle{Interaction Design for Computer Scientists}
\addauth{Steffan Sølvsten}{201505832@post.au.dk}{\, au534068}
\maketitle

\begin{abstract}
  \noindent After having to read three different books on Human Computer Interaction, this is an attempt to dispose of the frustrating amount of unecessary information and vague or non-existent definitions in the HCI universe apparent in all text books. This is to be a dense, clearly defined, and small guide to interaction design
\end{abstract}

\newpage
\tableofcontents

\newpage
%%%%%%%%%%%%%%%%%%%%%%%%%%%%%%%%%%%%%%%%%%%%%%%%%%%% 
% Chapter: Introduction
\section{Introduction}
\label{sec:introduction}

\chapter{Introduction} \label{chap:introduction}
Computer Scientists at Aarhus University tend to have a \emph{love-hate} relationship with all their HCI courses - with the only exception, that there is no \emph{love} at all. Many especially are frustratd due to the lack of a text book, which is clear and cuts to the point. All the books and other material, that have been used in the courses and can be found in the references, are drowning the actual points and what is important in huge blobs of useless text. The following paragraph from a 400 page long textbook is a prime example of this offense

\blockcquote[p. 435]{rogers}{\it What to evaluate ranges from low-tech prototypes to complete systems; a particular screen function to the whole workflow; and from aesthetic design to safety features. For example, developers of a new web browser may want to know if users find items faster with their product, whereas developers of an ambient display may be interested in whether it changes people's behaviour. Government authorities may ask if a computerized system for controlling traffic lights results in fewer accidents or if a website complies with the standards required for users with disabilities. Makers of a toy ask if 6-year-olds can manipulate the controls and whether they are engaged by its furry case, and whether the toy is safe for them to play with. A company that develops personal, digital music players may want to know if the size, color, and shape of the casings are liked by people from different age groups living in different countries. A software company may want to asses market reaction to its new homepage design.}

\noindent While the first sentence is useful and the next one arguably too, the next 4 are completely useless. If the reader had not already gotten the point by the first example or second, they would certainly not have learned it by the time they reached the end of this monstrosity of a paragraph. Similarly in the 27 pages long paper by Lim and Tenenberg \cite{lim} the same style of writing is evident, where the reader can scan the text and can already reach a conclusion two pages prior to the author. Simultaneously several methods and concepts are only very vaguely defined or not at all. For example a \emph{rich picture} is in the text book of Benyon only defined with a mere mention to its existence and a reference to a figure with two examples. \cite[p. 51-52]{benyon_14}

It is extremely saddening that something like this should result in students dismissing ever delving into this fascinating field. This document is an attempt to cut to the point by clearly defining all concepts, frameworks, and methods, compare them, and highlight their its pros and cons. We hope this can make the experience of HCI for some students less frustrating, maybe even turn it from an excruviating experience to a delightful one. This document is divided into three parts: theory, examples, and exercises. This way we wish that the theory will not drowned in examples. With the exercises we want to provide small and focused samples of data and problems, where the theory can be applied and that the material is learned effectively. With the theory seperated into its own section, we see this document first and foremost as a small encyclopedia of the world of HCI, yet we hope with the examples and exercises, that we can produce a text books, which can be used in a course in interaction design in its own right.

Every key word and concept is attempted to be categorized as one of the following
\begin{itemize}
   \item \textbf{Definition}: Definition of basic underlying vocabulary.

   \item \textbf{Concept}: An abstract idea and/or generalization of repeated human behaviour.

   \item \textbf{Framework}: An abstract conceptual structure, around which a lot of methods are designed.
  
   \item \textbf{Method}: A process to attain knowledge, understanding, and/or conclusions
  
   \item \textbf{Tool}: An entity, which is created to be directly used to inform the later design process. In a lot of ways this overlaps with a method, though the point to stress is that a physical or textbased entity is created to be directly used later, rather than just the conclusions.
\end{itemize}


\newpage
%%%%%%%%%%%%%%%%%%%%%%%%%%%%%%%%%%%%%%%%%%%%%%%%%%%%
% Chapter: Theory and Concepts
\section{Theory and Concepts}
\label{sec:1}

In this section all frameworks, concepts and methods will be defined, compared and criticised. They are simultaneously grouped into the point in development they are to be used. This grouping though makes it seem like a design is a linear process, which it is everything but. The design process should as much as possible follow the idea of agile development in concept \ref{conc:agile}. This means, that no section is ever properly finished, as the designer will keep on going back, but rather the next section is also juggled together with the prior things at the same time.
\begin{concept}[agile development] \label{conc:agile} \index{agile development} 
  
\end{concept}

\begin{framework}[Y-Model] \label{fw:y_model} \index{Y-Model}
  
\end{framework}
\subsection{Initial}
\label{sec:1_initial}


\begin{framework}[PACT] \label{fw:pact} \index{PACT Framework}
  
\end{framework}
\subsection{Data Gathering}
\label{sec:1_data_gathering}

\subsection{Data Analysis}
\label{sec:1_data_analysis}

\subsection{Envisionment}
\label{sec:1_envisionment}

\subsection{Prototyping}
\label{sec:1_prototyping}


\newpage
%%%%%%%%%%%%%%%%%%%%%%%%%%%%%%%%%%%%%%%%%%%%%%%%%%%% 
% Chapter: Examples
\section{Examples}
\label{sec:2}

This section consists of different real-life examples, which exemplify one or more concepts defined in section \ref{sec:1} and are hence also grouped into equivalent sections.
% TODO: Add sections
%       Need to know structure of part 1 first

\newpage
%%%%%%%%%%%%%%%%%%%%%%%%%%%%%%%%%%%%%%%%%%%%%%%%%%%% 
% Chapter: Exercises
\section{Exercises}
\label{sec:3}

This section consists of different exercises, such that you can try to apply the theory from section \ref{sec:1} in small isolated cases written up for maximizing your learning outcome. These exercises are grouped into subsections in relation to the subsections in \ref{sec:1}.
% TODO: Add sections
%       Need to know structure of part 1 first


\newpage
%%%%%%%%%%%%%%%%%%%%%%%%%%%%%%%%%%%%%%%%%%%%%%%%%%%% 
% References
\begin{minipage}{1.0\textwidth}
  \begin{thebibliography}{9}
  \bibitem{benyon_14}
    Benyon, David: \emph{Designing Interactive Systems}, Pearson, 3rd edition, 2014
  \bibitem{benyon_10}
    Benyon, David: \emph{Designing Interactive Systems}, Pearson, 2nd edition, 2010
  \bibitem{rogers}
    Rogers, Yvonne: \emph{Interaction Design: Beyond human-computer interaction}, Wiley, 3rd edition, 2011
  \bibitem{lim}
    Lim, Youn-Kyung and Tenenberg, Josh: \emph{The Anatomy of Prototypes}, ACM Transactions on Computer-Human Interaction, Vol. 15, 2008
  \end{thebibliography}
  \bibliographystyle{abbrv}
  \bibliography{referencer}
\end{minipage}

\newpage
%%%%%%%%%%%%%%%%%%%%%%%%%%%%%%%%%%%%%%%%%%%%%%%%%%%% 
% Index
\printindex

\end{document}