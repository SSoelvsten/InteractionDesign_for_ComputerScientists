%%%%%%%%%%%%%%%%%%%%%%%%%%%
% Set up document         %
%%%%%%%%%%%%%%%%%%%%%%%%%%%
\documentclass[a4, english, twoside]{article}

% Import SSoelvsten's LaTeX preamble
% github.com/SSoelvsten/LaTeX-Preamble_and_Examples
\usepackage{import}
\subimport{../LaTeX-Preamble_and_Examples/preamble/}{preamble_en.tex}

% New settings for this specific document

% Use these for 'definitions' in the theory section
\newtheorem{concept}[theorem]{Concept}
\newtheorem{framework}[theorem]{Framework}
\newtheorem{method}[theorem]{Method}
\newtheorem{tool}[theorem]{Tool}

% Indexing
% To index something, write \index{keyword}
\usepackage{makeidx}
\usepackage[columns=2]{idxlayout}
\makeindex

% Images in the background (Front page)
\usepackage[some,top]{background}

%%%%%%%%%%%%%%%%%%%%%%%%%
% Document starts here! %
%%%%%%%%%%%%%%%%%%%%%%%%%

\begin{document}
\settitle{\sc Interaction Design - In a Nutshell}
\addauth{Steffan Sølvsten}
%\date{}
%\maketitle

% Custom title
\thispagestyle{empty} % remove header and footer
\begin{center}
	\phantom{}\vspace{3.1cm}
	{\Huge\sc Interaction Design}
	\\ \vspace{1em}
	{\LARGE\sc In a Nutshell}
	\\
	\line(1,0){250}
	\\ \vspace{1em}
	{\LARGE Steffan Sølvsten}
\end{center}

\SetBgContents{
    \includegraphics[scale=0.055]{img/front.png}
}
\SetBgOpacity{1}
\BgThispage

\newpage
\thispagestyle{empty} % remove header and footer
\noindent \textbf{Steffan Sølvsten}

Aarhus, Denmark

\href{mailto:steffan.soelvsten@hotmail.com}{steffan.soelvsten@hotmail.com}

\vspace{1em}
\noindent Project started 2018

\vspace{1em}
\noindent This document and the sources from which it has been compiled are in
the public domain under the GNU General Public License 3.0. In short you may
distribute and change this document, but the result must remain open source and
free.

\begin{center}
  \href{https://www.gnu.org/licenses/gpl-3.0.en.html}{www.gnu.org/licenses/gpl-3.0.en.html}
\end{center}

\vspace{2em}
% https://tex.stackexchange.com/questions/128636/center-hrule-in-the-middle-of-the-page
\noindent\hfil\rule{0.8\textwidth}{.4pt}\hfil

\vspace{2em}
\noindent This book has been created as a joined group effort of the students of Aarhus
University. Thanks to everyone who has contributed with smaller og bigger
sections, corrections, feedback, proof-reading and much more.
  
\begin{multicols}{2}
  Johannes Ernstsen

  \hfill
  \columnbreak
  
  Kira Kutscher
\end{multicols}

\newpage
\thispagestyle{empty} % remove header and footer
\tableofcontents

\newpage
%%%%%%%%%%%%%%%%%%%%%%%%%%%%%%%%%%%%%%%%%%%%%%%%%%%%
% Chapter: Introduction
\setcounter{page}{1}

\section{Introduction}
\label{sec:introduction}

\chapter{Introduction} \label{chap:introduction}
Computer Scientists at Aarhus University tend to have a \emph{love-hate} relationship with all their HCI courses - with the only exception, that there is no \emph{love} at all. Many especially are frustratd due to the lack of a text book, which is clear and cuts to the point. All the books and other material, that have been used in the courses and can be found in the references, are drowning the actual points and what is important in huge blobs of useless text. The following paragraph from a 400 page long textbook is a prime example of this offense

\blockcquote[p. 435]{rogers}{\it What to evaluate ranges from low-tech prototypes to complete systems; a particular screen function to the whole workflow; and from aesthetic design to safety features. For example, developers of a new web browser may want to know if users find items faster with their product, whereas developers of an ambient display may be interested in whether it changes people's behaviour. Government authorities may ask if a computerized system for controlling traffic lights results in fewer accidents or if a website complies with the standards required for users with disabilities. Makers of a toy ask if 6-year-olds can manipulate the controls and whether they are engaged by its furry case, and whether the toy is safe for them to play with. A company that develops personal, digital music players may want to know if the size, color, and shape of the casings are liked by people from different age groups living in different countries. A software company may want to asses market reaction to its new homepage design.}

\noindent While the first sentence is useful and the next one arguably too, the next 4 are completely useless. If the reader had not already gotten the point by the first example or second, they would certainly not have learned it by the time they reached the end of this monstrosity of a paragraph. Similarly in the 27 pages long paper by Lim and Tenenberg \cite{lim} the same style of writing is evident, where the reader can scan the text and can already reach a conclusion two pages prior to the author. Simultaneously several methods and concepts are only very vaguely defined or not at all. For example a \emph{rich picture} is in the text book of Benyon only defined with a mere mention to its existence and a reference to a figure with two examples. \cite[p. 51-52]{benyon_14}

It is extremely saddening that something like this should result in students dismissing ever delving into this fascinating field. This document is an attempt to cut to the point by clearly defining all concepts, frameworks, and methods, compare them, and highlight their its pros and cons. We hope this can make the experience of HCI for some students less frustrating, maybe even turn it from an excruviating experience to a delightful one. This document is divided into three parts: theory, examples, and exercises. This way we wish that the theory will not drowned in examples. With the exercises we want to provide small and focused samples of data and problems, where the theory can be applied and that the material is learned effectively. With the theory seperated into its own section, we see this document first and foremost as a small encyclopedia of the world of HCI, yet we hope with the examples and exercises, that we can produce a text books, which can be used in a course in interaction design in its own right.

Every key word and concept is attempted to be categorized as one of the following
\begin{itemize}
   \item \textbf{Definition}: Definition of basic underlying vocabulary.

   \item \textbf{Concept}: An abstract idea and/or generalization of repeated human behaviour.

   \item \textbf{Framework}: An abstract conceptual structure, around which a lot of methods are designed.
  
   \item \textbf{Method}: A process to attain knowledge, understanding, and/or conclusions
  
   \item \textbf{Tool}: An entity, which is created to be directly used to inform the later design process. In a lot of ways this overlaps with a method, though the point to stress is that a physical or textbased entity is created to be directly used later, rather than just the conclusions.
\end{itemize}


\newpage
%%%%%%%%%%%%%%%%%%%%%%%%%%%%%%%%%%%%%%%%%%%%%%%%%%%%
% Chapter: Theory and Concepts
\section{Theory}
\label{sec:1}

In this section all frameworks, concepts and methods will be defined, compared and criticised. They are simultaneously grouped into the point in development they are to be used.

Today design of interactive systems is centered around the end user and constant reevaluation, which the Y-Model in framework \ref{fw:y_model} reflects perfectly.

\begin{framework}[Y-Model] \label{fw:y_model} \index{Y-Model}
  
\end{framework}


\subsection{Underlying Concepts}
\label{sec:1_concepts}
This section contains theory and concepts underlying all the following sections.

\begin{definition}[Stakeholder] \label{def:stakeholder} \index{Stakeholder}
  Any person, who is affected by the system to be or being designed - both
  the potential users, but also the organization and people, where the use of
  the system will make a difference to them.
  \cite[p. 50]{benyon_14}
\end{definition}

\begin{definition}[Tacit Knowledge] \label{def:tatic_knowledge} \index{Tacit Knowledge}
  Knowledge a person has and uses, but is unable to put in words or in other ways explain or pass on to others.
\end{definition}


\subsection{Agile Development}
\label{sec:1_agile}


\begin{concept}[Agile Method] \label{conc:agile} \index{Agile Method}
  
\end{concept}
  
\begin{framework}[Scrum] \label{fw:scrum} \index{Scrum}
  
\end{framework}

\subsection{Stages of design}
The grouping of methods, concepts and more makes it seem like the design process is a linear process, which it should not be at all. The design process should as much as possible follow the idea of agile development in section \ref{sec:1_agile}. This means, that no section is ever properly finished, as the designer will keep on going back, but rather the next section is also juggled together with the prior things at the same time.

\subsubsection{Initial}
\label{sec:1_initial}

\begin{framework}[PACT] \label{fw:pact} \index{PACT Framework}
  
\end{framework}
\subsubsection{Data Gathering}
\label{sec:1_data_gathering}

\begin{method}[Interview] \label{meth:interview} \index{Interview}
  Directly talking and asking questions to stakeholders is one key way to learn
  of their needs, requirements and problems. 
  
  Interviews can be done at several stages of the process: during the data
  analysis or getting evaluation of a the data analysis, envisionment or on a
  prototype. Interviews can take on many different forms, which all have different pros and
  cons.
  \cite[p. 142-146]{benyon_14}
\end{method}

\subparagraph{Structured Interview / Survey} \index{Structured Interview} \index{Survey}
A fully prestructured interview, where the interviewee only answers premade
questions. While unable to follow up on unexpected answers, they are easy to
carry out, which creates a high-quantity of responses, which can be analysed
statistically. This can also be done as a survey with the same outcome. 
\cite[p. 142]{benyon_14}

\subparagraph{Semistructured Interview} \index{Semistructured Interview}
An interview with prepared questions and topics to be asked and covered, while
the interviewer still allows for the discussion to digress in relevant
directions. Preperations mainly consist of creating a list of topics and prompts
to cover, and while more complicated for the interviewer, the data is more
nuanced and of higher-quality.
\cite[p. 143]{beynon_14}

\subparagraph{Unstructured Interview} \index{Unstructured Interview}
The interviewer makes no preperation going into the interview, either because
little information of the subjectmatter can be researched, or to minimize any
biases and keep the interviewer open to the interviewee's answers.
\cite[p. 143]{benyon_14}

\begin{method}[Observation] \label{meth:observation} \index{Observation}
  
\end{method}

\begin{method}[Contextual Interview] \label{meth:contextual_interview} \index{Contextual Inteview}
  
\end{method}

\begin{definition}[Artifact] \label{def:artifact} \index{Artifact}
  
\end{definition}

\begin{method}[Artifact Gathering] \label{meth:artifact_collection} \index{Artifact Collection}
  Collecting artifacts either physically or by taking picture of them, if you are unable to retreive the artifact from the site.
\end{method}
\subsubsection{Data Analysis}
\label{sec:1_data_analysis}
The main purpose of the data analysis step is to take the data gathered and extract conclusions about the system to be designed. After having done the analysis, criteria for the system to be designed will be clear to the designer. Furthermore in this step a lot of tools such as personas and user stories are created, which are valuable tools in the later design process.

Since always more knowledge is gained around the field for which you design, it is normal to turn back to this stage quite a lot to redo or add to prior work.

\begin{method}[Affinity Diagram] \label{meth:affinity_diagram} \index{Affinity Diagram}
  
\end{method}

\begin{method}[Rich Picture] \label{meth:rich_picture} \index{Rich Picture}
  
\end{method}

\begin{tool}[Persona] \label{tool:persona} \index{Persona}
  
\end{tool}

\begin{tool}[User Stories (Scenario)] \label{tool:user_stories} \index{User Stories (Scenario)}
  
\end{tool}

\begin{figure}
  \centering
  % TODO : Recreate the figure from Benyon
  \caption{The four different types of scenarios \cite[p. 67]{benyon_14}}
  \label{fig:scenarios}
\end{figure}

\begin{definition}[Flow Model] \label{meth:flow_model} \index{Flow Model}
  
\end{definition}

\begin{definition}[Sequence Model] \label{meth:sequence_model} \index{Sequence Model}
  
\end{definition}

\subsubsection{Envisionment}
\label{sec:1_envisionment}

\begin{method}[Individual Snapshot] \label{meth:individual_snapshot} \index{Individual Snapshot}
  
\end{method}

\begin{tool}[Storyboard] \label{meth:storyboard} \index{Storybaord}
  
\end{tool}

\begin{tool}[Conceptual Scenario] \label{meth:conceptual_scenario} \index{Conceptual Scenario}
  This is based on Benyon
\end{tool}

\begin{tool}[Conceptual Model] \label{def:conceptual_model} \index{Conceptual Model}
  
\end{tool}

\begin{definition}[Metaphor] \label{def:metaphor} \index{Methaphor}
  
\end{definition}

\begin{definition}[Design Language] \label{def:design_language} \index{Design language}
  
\end{definition}

\begin{tool}[Navigation Map] \label{meth:navigation_map} \index{Navigation Map}
  
\end{tool}

\begin{tool}[Mood Board] \label{tool:mood_board} \index{Mood Board}
  
\end{tool}


\subsubsection{Prototyping}
\label{sec:1_prototyping}
Prototypes as filters... \cite{lim}


\begin{method}[Concrete Scenario] \label{meth:concrete_scenario} \index{Concrete Scenario}
  
\end{method}

\begin{method}[Use Cases (Scenario)] \label{meth:use_cases} \index{Use Cases (Scenario)}
  
\end{method}

\begin{definition}[Lo-Fi Prototype] \label{def:lo-fi_prototype} \index{Lo-Fi Prototype}
  
\end{definition}

\begin{definition}[Hi-Fi Prototype] \label{def:hi-fi_prototype} \index{Hi-Fi Prototype}
  
\end{definition}

\begin{method}[Evaluation of prototypes] \label{meth:evaluation_of_prototypes} \index{Evaluation of Prototypes}
  
\end{method}

\newpage
%%%%%%%%%%%%%%%%%%%%%%%%%%%%%%%%%%%%%%%%%%%%%%%%%%%% 
% Chapter: Examples
\section{Examples}
\label{sec:2}

This section consists of different real-life examples, which exemplify one or more concepts defined in section \ref{sec:1} and are hence also grouped into equivalent sections.
% TODO: Add sections
%       Need to know structure of part 1 first

\newpage
%%%%%%%%%%%%%%%%%%%%%%%%%%%%%%%%%%%%%%%%%%%%%%%%%%%% 
% Chapter: Exercises
\section{Exercises}
\label{sec:3}

This section consists of different exercises, such that you can try to apply the theory from section \ref{sec:1} in small isolated cases written up for maximizing your learning outcome. These exercises are grouped into subsections in relation to the subsections in \ref{sec:1}.
% TODO: Add sections
%       Need to know structure of part 1 first


\newpage
%%%%%%%%%%%%%%%%%%%%%%%%%%%%%%%%%%%%%%%%%%%%%%%%%%%% 
% References
\begin{minipage}{1.0\textwidth}
  \begin{thebibliography}{9}
  \bibitem{benyon_14}
    Benyon, David: \emph{Designing Interactive Systems}, Pearson, 3rd edition, 2014
  \bibitem{benyon_10}
    Benyon, David: \emph{Designing Interactive Systems}, Pearson, 2nd edition, 2010
  \bibitem{rogers}
    Rogers, Yvonne: \emph{Interaction Design: Beyond human-computer interaction}, Wiley, 3rd edition, 2011
  \bibitem{lim}
    Lim, Youn-Kyung and Tenenberg, Josh: \emph{The Anatomy of Prototypes}, ACM Transactions on Computer-Human Interaction, Vol. 15, 2008
  \end{thebibliography}
  \bibliographystyle{abbrv}
  \bibliography{referencer}
\end{minipage}

\newpage
%%%%%%%%%%%%%%%%%%%%%%%%%%%%%%%%%%%%%%%%%%%%%%%%%%%% 
% Index
\printindex

\newpage
%%%%%%%%%%%%%%%%%%%%%%%%%%%%%%%%%%%%%%%%%%%%%%%%%%%% 
% Backcover
\thispagestyle{empty} % remove header and footer
\begin{center}
  \begin{minipage}[r]{0.6\linewidth}
    \phantom{}\vspace{2.7cm}
    \begin{abstract}
      \noindent After having to read three different books on Human Computer Interaction, this is an attempt to dispose of the frustrating amount of unecessary information and vague or non-existent definitions in the HCI universe apparent in all these text books. This is to be a dense, clearly defined, and small guide to interaction design
    \end{abstract}  
  \end{minipage}  
\end{center}

\SetBgContents{
    \includegraphics[scale=0.055]{img/front.png}
}
\SetBgOpacity{1}
\BgThispage

\end{document}
