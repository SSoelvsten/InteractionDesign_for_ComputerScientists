\begin{model}[Flow Model] \label{mod:flow_model} \index{Flow Model}
  Based on a specific point of view a flow model visualizes the flow between individuals and groups in a context. This highlights the communcation and the coordination between individuals and groups, and the breakdowns possible herein. \cite[p. 278]{benyon_10}
  
  A flow model contains the following components \cite[p. 278-279]{benyon_10}
  \begin{itemize}
  	\item \emph{Individuals} themselves, meaning more than one individual with the same role can exist. These are not labelled with names, but rather a generic identifier.
  	\item \emph{Responsibilities} of the individuals
  	\item \emph{Groups}, being more than one individual with the same of responsibilities. All interactions between the outside and individuals inside the group are indepdent in nature of the specific individual involved.
  	\item \emph{Flow} of the communcation between individuals and groups in order to get their work done, being both the content and the medium of the conversation.
  	\item \emph{Artefacts} (model \ref{mod:artefact}), also including important conclusions from conversations
  	\item \emph{Topic of communication} or \emph{Action}
  	\item \emph{Places}, if they are central to the coordination of work
  	\item \emph{Breakdowns} in communication and coordination
  \end{itemize}
  A flow model is created through the following steps \cite[p. 279-281]{benyon_10}
  \begin{enumerate}
  	\item Choose a point of view and add individuals and groups in bubbles and add their responsibilities
  	\item Add locations and shared information. Further, add the flow between individuals and groups as arrows together the arrow labeled with the topics and artefacts part of the interaction. Artefacts are visualized as boxes either at the end of the arrow or on the arrow.
  	\item Add breakdowns as red flashes with a short description of the cause and nature of the problem.
  \end{enumerate}
\end{model}
