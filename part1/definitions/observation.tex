\begin{method}[Observation] \label{meth:observation} \index{Observation}
  The designer only observes the user's behaviour without physically or verbally interacting with the subject. This works especially well to uncover the user's tacit knowledge (definition \ref{def:tacit_knowledge}). In the the Y-model, observations can be used in the field, observing subjects everyday behaviour, as part of the \emph{understanding}-phase; and they can be used in controlled environments, observing the subjects attempting to solve a predefined task, as part of the \emph{evaluation}-phase. \cite[p.247-248]{rogers}

  The following are a set of bulletpoints for the designer to answer the questions they might have \cite[p.249]{rogers}
  \begin{itemize}
    \item \emph{Space}: The layout and contents of the environment.
    \item \emph{Actors}: Relevant details of the people involved.
    \item \emph{Goals}: The goals of the actors.
    \item \emph{Activities}: What people are doing.
    \item \emph{Artefacts}: Artefacts used throughout (model \ref{mod:artefact}).
    \item \emph{Time}: The order of the activities and events.
    \item \emph{Context}: The context the activities observed take place within.
    \item \emph{Feelings}: The mood of the actors individually and as a group.
  \end{itemize}
\end{method}