\begin{method}[Affinity Diagram] \label{meth:affinity_diagram} \index{Affinity Diagram}
  By use of a brainstorm and Post-it notes the users requirements, wishes, and needs are filtered out of the data collected. The method is effective and especially great early in the design process to cut through the massive amount of information and the complexity of the domain. The construction proceeds as follows
  \begin{itemize}
    \item In a brainstorm the designers write down any requirements, wishes, needs etc. on Post-it notes. At most one sentence is put on a single Post-it.
    \item The brainstorm is stopped when the designers are unable to think of more or they reach a few hundred Post-it notes.
    \item On a board or a wall all Post-its are grouped by common themes and duplicates are thrown away.
    \item Headlines for every group is made and the content on the board is recorded
  \end{itemize}
  From the groupings features, properties and expected behaviours of the system to be designed can start to come forth. \cite[p. 299-300]{benyon_10}
\end{method}