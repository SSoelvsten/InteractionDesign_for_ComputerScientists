\begin{model}[Artefact] \label{mod:artefact} \index{Artefact}
  Physical objects used by people as part of their work or as part of their physical environment. These items can give an insight into the current workflow of the users and the problems and breakdowns that they experience. Later in the design process artefacts can be inspiration for what media can be used, what data to include and what pitfalls to circumvent. \cite[p. 285]{benyon_10}

  The following is a list of what to look for in an artefact \cite[p. 285]{benyon_10}
  \begin{itemize}
    \item The \emph{information} within the artefact
    \item The \emph{structure} of the artefact, i.e. the placement and different parts of the artefact and for whom each is intended
    \item Informal \emph{annotations}, where the user has manually enhanced the artefact with handwriting or post-it notes due to the artefact insufficiently solves its domain.
    \item The \emph{presentation}, such as color, fonts etc., showing the overall feel, values conveyed and importance of its parts.
    \item The \emph{temporal} changes during use
    \item \emph{When}, for \emph{what}, and by \emph{whom} it is used
    \item Any \emph{breakdowns} evident
  \end{itemize}
\end{model}