\begin{definition}[4 types of interaction] \label{def:4_types_of_interaction} \index{4 types of interaction} 
  We differentiate between four types of interaction, which are not mutually exclusive nor definitive. \cite[p. 47]{rogers}
  \begin{itemize}
    \item Instructing: \index{Instructing interaction}
      Users issue instructions to the system through buttons, function keys, gesture or by voice or similar. This type of interaction is very efficient when first learned, but it requires a substantial learning-effort.
      
    \item Conversing: \index{Conversing interaction}
      Either by voice or text users interact with the system through a dialog. Users are good at human languages, but it can lead to misunderstanding and it is also very ineffective..
      
    \item Manipulating: \index{Manipulating interaction}
      Users interact by manipulating objects in a virtual or physical space. This is familiar and can offer a higher level of control, but it is limited by the metaphors (definition \ref{def:metaphor}) and it can be ineffective. It is governed by the following three principles
      \begin{itemize}
        \item Objects of interest should be continuously represented.
        \item Should consist of small and incremental actions with immediate feedback
        \item Physical manipualtion with buttons should be used rather than instructions.
      \end{itemize}
      
    \item Exploring:  \index{Exploring interaction}
      An interaction designed around the user's navigation through the system in a physical or visual space. This is more engaging and free, but very inefficient.
  \end{itemize}
\end{definition}