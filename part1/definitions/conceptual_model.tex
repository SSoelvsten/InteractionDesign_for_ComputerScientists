\begin{concept}[Conceptual Model] \label{conc:conceptual_model} \index{Conceptual!Model}
  An abstracted description of concepts users need to both understand and operate the system. A conceptual model consists of the following
  \begin{itemize}
    \item Metaphors of definition \ref{def:metaphor}
    \item Concepts used in the system, such as task-domain objects, their attributes and the users's interactions with them. 
      \begin{itemize}
      \item The relationsship between these concepts in terms of their grouping, inclusion of one in another, and their mutual importance
      \item How these concepts are mapped to the user experience intended to support
      \end{itemize}
  \end{itemize}
  The conceptual model allows designers to early on discuss how well the main concepts are supported by a design. Designers can with this evaluate different kinds of interactions, the various possible procedures to do an activity, and the choice of metaphors. This way they can decide on a good core-design before they begin work on more concrete ideas. \cite[p. 40-41]{rogers}
\end{concept}
