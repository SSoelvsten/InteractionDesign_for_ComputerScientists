\begin{method}[Interview] \label{meth:interview} \index{Interview}
  Directly talking and asking questions to stakeholders is one key way to learn
  of their needs, requirements and problems. 
  
  Interviews can be done at several stages of the process: during the data
  analysis or getting evaluation of a the data analysis, envisionment or on a
  prototype. Interviews can take on many different forms, which all have different pros and
  cons.
  \cite[p. 142-146]{benyon_14}
\end{method}

\subparagraph{Structured Interview / Survey} \index{Structured Interview} \index{Survey}
A fully prestructured interview, where the interviewee only answers premade
questions. While unable to follow up on unexpected answers, they are easy to
carry out, which creates a high-quantity of responses, which can be analysed
statistically. This can also be done as a survey with the same outcome. 
\cite[p. 142]{benyon_14}

\subparagraph{Semistructured Interview} \index{Semistructured Interview}
An interview with prepared questions and topics to be asked and covered, while
the interviewer still allows for the discussion to digress in relevant
directions. Preperations mainly consist of creating a list of topics and prompts
to cover, and while more complicated for the interviewer, the data is more
nuanced and of higher-quality.
\cite[p. 143]{benyon_14}

\subparagraph{Unstructured Interview} \index{Unstructured Interview}
The interviewer makes no preperation going into the interview, either because
little information of the subjectmatter can be researched, or to minimize any
biases and keep the interviewer open to the interviewee's answers.
\cite[p. 143]{benyon_14}
