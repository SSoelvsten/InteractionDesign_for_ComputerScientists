\begin{method}[Contextual Interview] \label{meth:contextual_interview} \index{Contextual Inteview}
  In its most basic form \emph{Contextual Interview} is the combination of interviewing the users (method \ref{meth:interview}) while also observing them (method \ref{meth:observation}) in the context of their work. When the analyst asks the subject their questions is highly dependant on the situation the contextual interview is conducted within, but the power of the contextual interview lies in the analyst being able to ask the subject about their actions and get the subject's feedback on the analyst's interpretations. \cite[p. 273-276]{benyon10}

  A typical contextual interview could look as follows
  \begin{itemize}
    \item 15 minutes: Introductions, permissions, and explaining the contextual interview's focus
    \item 2 to 3 hours: Observations intermixed with conversation and questions
    \item 15 minutes: The analyst reviews their reflections with the subject
  \end{itemize}
  though if the activities observed would be nonrepresentative if the analyst intervened with questions, then it makes more sense to ask the subject about their activities and thoughtprocess after finishing the activity or the whole observation. \cite[p. 275-276]{benyon10}  
\end{method}
