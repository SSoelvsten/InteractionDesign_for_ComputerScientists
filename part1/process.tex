\chapter{Process of human-centred design} \label{chap:process}
The key for modern design of human-computer interaction is to take the human, their needs and abilities, as a starting point. A designsolution unusable by the intended users is a failure the designer is responsible for. Hence, the core part of successful human-centred design is to constantly evaluate everything, preferably with the stakeholders.

But why even bother going through all the hazzle and ressources and instead just skip everything contained in this book and just make a product? By having the human central in the design will create a product more likely to be widely adopted and used longer, which likely gives a big return on investment. Furthermore several pitfalls in terms of safety and ethics can be circumvented, if the design is human-centred. Furthermore the design of technologies need to accomodate and support human values. \cite[p. 20-22]{benyon_14}

\section{Y-model} \label{sec:y_model}
\begin{framework}[Y-Model] \label{fw:y_model} \index{Y-Model}
  Characterization of the overall design process into the following four activities
  \begin{itemize}
    \item \emph{Evaluation}:
      Everything produced by the designer, be it understanding of the users needs and context, abstract design ideas or concrete prototypes, has to be evaluated with the end users. The type of evaluation varies heavily based on what is evaluated and the focus of the designer.
    \item \emph{Understanding}:
      Research of stakeholders (definition \ref{def:stakeholder}) and their activities and the contexts within the problem domain. On an abstract level the designer wants to identify the stakeholders goals, requirements, wishes and needs, together with the restrictions on technologies and people due to the surrounding context.
    \item \emph{Design}:
      Any part of the design, which are not immediately resulting in a physical product, but rather on a more abstract level specifies the product.
      \begin{itemize}
        \item \emph{Conceptual Design}:
          On an abtract level the overall purpose of the system, being its functionality and the information provided by it, is specified together with finding a conceptual model (concept \ref{conc:conceptual_model}) to clearly communicate the design to the user. This can be manifested in a specification and vision of the system or through other ways.
          
        \item \emph{Physical Design}:
          Going from the abstract to the concrete, the designer specifies the sequence of interactions, the presentation of information and functions, and the physical feel of the system. It consists of the following three components
          \begin{itemize}
            \item \emph{Operational Design}:
              The way functions and information is structured relative to each other throughout the system
              
            \item \emph{Representational Design}:
              The overall feel and look of the design
              
            \item \emph{Interaction Design}:
              The structure and sequence of the interactions between humans and the system.
              
          \end{itemize}
      \end{itemize}

    \item \emph{Envisionment}:
      Making the design ideas real and bringing them physically into the world, the designers create everything from simple sketches to full blown high-fidelity prototypes (definition \ref{def:hi-fi_prototype}). With these in hand the designers can concretely evaluate their ideas with eachother and with the users.
      
  \end{itemize}
  As visualized in figure \ref{fig:y_model} the design process is non-linear and evaluation is central to the process. The designer may start in any of the four activities and everything by the designer is evaluated before moving onto the next activity. \cite[p. 49-54]{benyon_14}
\end{framework}

\begin{figure}[ht!]
  \centering
  \begin{tikzpicture}
    \node[cloud, draw, aspect=2] (evaluation) {Evaluation};
    \node[cloud, draw, aspect=2] (envisionment) [above left=1cm of evaluation] {Envisionment};
    \node[cloud, draw, aspect=2] (understanding) [above right=1cm of evaluation] {Understanding};
    \node[cloud, draw, aspect=2, align=left] (design) [below =1cm of evaluation] {Design \\ \\ \\};
    \node[cloud, draw, fill=white, aspect=2, align=right] (conceptual) [below left =-1.7cm of design] {Conceptual \\ Design};
    \node[cloud, draw, fill=white, aspect=2, align=left] (physical) [below right =-1.7cm of design] {Physical \\ Design};
    \path[<->, line width=0.6mm]
        (evaluation) edge (envisionment)
                     edge (understanding)
                     edge (design)
    ;
  \end{tikzpicture}
  \caption{The Y-model visualized \cite[p. 49]{benyon_14}}
  \label{fig:y_model}
\end{figure}


\section{PACT} \label{sec:pact} \index{PACT}
\begin{figure}
  \centering
  TODO
  \caption{The mutual influence of technologies and activities \cite[p. 26]{benyon_14}}
  \label{fig:activities_and_technologies}
\end{figure}

To make design human-centred we want to consider the \emph{people} involved, the \emph{activities} in which they are involved, the surrounding \emph{context}, and the \emph{technologies} involved. This framework proposed, furthermore has to account for the activities dictating the requirements of the technologies designed, but furthermore that technologies designed have unforseen impact on the nature of the activities, as shown in figure \ref{fig:activities_and_technologies} \cite[p. 25-26]{benyon14}

In a PACT analysis the designer evaluates the many dimensions of \emph{P}s, \emph{A}s, \emph{C}s, and \emph{T}s possible and likely. On a current designsolution this can highlight designmistakes, while for new systems to be designed it can inform many design choices later, where it can be a foundation for the personas and scenarios of section \ref{sec:scenario_based_design}.

\paragraph{People}
The following are five different dimensions in which humans differ from eachother relevant to the design of solutions. \cite[p. 27-30]{benyon14}

\noindent
\begin{longtable}{rc>{\raggedright\arraybackslash}p{.66\textwidth}}
  Physical Differences & : &
  The humans senses, dexterity and disabilities, such as color blindness.
  \\
  Ergonomics & : &
  The \emph{ambient} and \emph{working} environment of definition \ref{def:ambient_environment} and \ref{def:working_environment} together with the physiology of the people in question.
  \\
  Psychological Differences & : &
  The spatial abilities, language, culture, attentionspan and memory of the people in the domain
  \\
  Mental models & : &
  The existence, detail and accuracy of the conceptual models (section \ref{sec:conceptual_model}) people have of the current systems and potentially useful other systems.
  \\
  Social Differences & : &
  The difference within the group of users or potential users, such as the amount of experience they have with the system and if it is necessary or not of them to use the system. Looking at this should be determined if the group is homogenous or heterogenous. 
\end{longtable}

\paragraph{Activities}
First the overall goal of the activities should be determined by the designer and then the details of the activities listed below examined  \cite[p. 33-35]{benyon14}

\noindent
\begin{longtable}{rc>{\raggedright\arraybackslash}p{.69\textwidth}}
  Temporal Aspects
  &  1 & How frequent or infrequent a system is used, since a lot of learning is by repetition.
  \\
  &  2 & How busy the user is, i.e. the pressure(s) under which the user has to use the system.
  \\
  &  3 & Is it continuous or interruptable? Should/can the system help the user continue the activity?
  \\
  &  4 & The response time of the system. Humans normally expect a $100$ms response time.
  \\
  Cooperation
  &  5 & Executed alone or as a group. Does the system have to support or work around communication and coordination?
  \\
  Complexity
  &  6 & Well defined tasks can easily be supported by a linear design, while vaguely defined tasks need a more complex non-linear design
  \\
  Safety-critical
  &  7 & How critical the consequences of a mistake is. The more critical the more important it is for the designer to prevent errors from occuring.
  \\
  &  8 & What happens and will people do on an error? Is it possible to recover?
  \\
  Nature of the content
  &  9 & The type and amount of data the user has to put in.
  \\
  & 10 & The type and amount of data the system has to convey to the user.
\end{longtable}

\paragraph{Context}
All activities happen in some context and are influenced by them. We differentiate between the following three types of contexts, where one or more can apply to a single activity. \cite[p. 34-35]{benyon14}

\noindent
\begin{longtable}{r>{\raggedright\arraybackslash}p{.69\textwidth}}
  Physical Environment &
  If it is noisy, cold, wet, dirty, with direct sunlight etc. and also technical properties, such as the quality of the internet connection.
  \\
  Social Context &
  The ability to get support from other people or tutorials, and the privacy issues to consider. Furthermore, this also is  also the social norms in play, which make some design solutions acceptable when alone but not while around others. 
  \\
  Organizational context &
  For the many teams in an organization, the organizational context can be divided into three categories \cite[p. 286]{doolen}
  \begin{itemize}
    \item Management Processs:
      The processes used by leaders related to organizational objectives, such as planning, goal setting in and across teams, and resource allocation.

    \item Organizational Culture:
      The shared values, beliefs and behavioral norms related to the work done. Especially important are coordination and cooperation between teams.
      
    \item Organizational Systems:
      Processes and other arrangements provided by the human ressources, such as feedback, recognition, training, education, and information.
  \end{itemize}
  % Technology -> organizational relations and deskilling
  % time, place etc.
\end{longtable}

\newpage
\paragraph{Technologies}
Humans interact with hardware and software through input and output of data, where the types of interactions possible are also dictated by the medium and the technology. The following characteristics of technology is to be considered. \cite[p. 36-43]{benyon14}

\noindent
\begin{longtable}{rc>{\raggedright\arraybackslash}p{.71\textwidth}}
  Material & : &
  Plastic, metal, glass, wood etc.
  \\
  Input & : &
  For example buttons, touch, mouse, joystick, wii-remote, speech, QR-codes etc.
  \\
  Output & : &
  All types of output are for vision, hearing and/or touch. For example display devices, sound (incl. speech), printer, haptics,  etc.
  \\
  Communication & : &
  If it is wired or wireless and what technology is used for sending data.
  \\
  Content & : &
  The type and the form of the data represented, which should be accurate, relevant and well presented. The content and amount of information changes the type of input and output reasonable to use
\end{longtable}
\section{Scenario based design} \label{sec:scenario_based_design}
Useful in all four stages of the Y-model (section \ref{sec:y_model}), scenarios and personas are one of the most fundamental and popular techniques to designing interactive systems. \cite[p. 62]{benyon14}

\begin{tool}[User Story (Scenario)] \label{tool:user_story} \index{User Story (Scenario)}
  A real-world rendition of the activities, experiences, knowledge etc. of a subject. It is of very high detail, and it can be recorded in many different formats, such as video, text, interview and much more. This is useful to gain an \emph{understanding} of the stakeholders needs and more. \cite[p. 62-63]{benyon_14}
\end{tool}

\begin{tool}[Conceptual Scenario] \label{tool:conceptual_scenario} \index{Conceptual Scenario}
  An abstract description of the users and the activities that the system supports. This high form of abstraction is great for imagining solutions in the \emph{Conceptual Design} phase and to condense knowledge in the \emph{understanding} phase (framework \ref{fw:y_model}). Furthermore the level of abstraction defines a domain of reasonable designsolutions without rejecting any reasonable designideas early, which is of great value later in the process. \cite[p. 63]{benyon_14}
\end{tool}

\begin{tool}[Concrete Scenario] \label{tool:concrete_scenario} \index{Concrete Scenario}
  A concretization of how a designsolution would work within a specific context and situation as part of the \emph{Conceptual Design}. These to a greater or lesser extend further define interface designs and the different types of and relationship between functions in the system. \cite[p. 64]{benyon_14}
\end{tool}
\begin{method}[Use Cases (Scenario)] \label{meth:use_cases} \index{Use Cases (Scenario)}
  
\end{method}


A central tool used in the scenario based design is also the \emph{persona}, which can especially enhance both concrete scenarios and use cases and in parts also conceptual scenarios and for revising user stories

\begin{tool}[Persona] \label{tool:persona} \index{Persona}
  A fully fleshed out and concrete characterization of a type of user designed for, including most importantly the persona's background, prerequisites, and goals, but also especially their human characteristics. The more detailed these are the easier they make it for the designer to emphasize with them and as a result design for the actual user. \cite[p. 55]{benyon_14}
\end{tool}

With all four types of scenarios we see all activities of the Y-model are covered. Many user stories show the users needs of the system, abstracted into very few conceptual scenarios defining the vision for the solution. These few then create many concrete scenarios generating concrete solutions, from them many more use cases specifying the exact details of the functions of the system. This relation between the stories, the design process and the Y-model is visualized in figure \ref{fig:scenarios}. \cite[p. 66, 196]{benyon14}

\begin{figure}
  \centering
  \begin{tikzpicture}
    % Scenarios
    \node[rectangle, fill=GreenYellow, draw, aspect=2] (user_stories) {User Stories \vphantom{p}};
    \node[rectangle, fill=GreenYellow, draw, aspect=2, align=right] (conceptual_scenarios) [right=1.7cm of user_stories] {Conceptual \\ Scenario};
    \node[rectangle, fill=GreenYellow, draw, aspect=2, align=left] (concrete_scenarios) [right=1.7cm of conceptual_scenarios] {Concrete \\ Scenario};
    \node[rectangle, fill=GreenYellow, draw, aspect=2] (use_cases) [right=1.7cm of concrete_scenarios] {Use Cases \vphantom{p}};
    
    % Lines between scenarios
    \path[-, line width=0.3mm]
        (user_stories) edge node [above] {$* \quad \quad 1$} (conceptual_scenarios)
        (conceptual_scenarios) edge node [above]  {$1 \quad \quad *$} (concrete_scenarios)
        (concrete_scenarios) edge node [above]  {$* \quad \quad *$} (use_cases)
    ;
    
    % Y-model
    \node[cloud, draw, aspect=2] (envisionment) [below =1.2cm of concrete_scenarios] {Envisionment};
    \node[cloud, draw, aspect=2] (understanding) [left =2.1cm of envisionment] {Understanding};
    \node[cloud, draw, fill=white, aspect=2, align=left] (physical) [below  =4.8cm of use_cases] {Physical \\ Design};
    \node[cloud, draw, fill=white, aspect=2, align=right] (conceptual) [left =3.7cm of physical] {Conceptual \\ Design $\quad$};
    
    % Text nodes
    \node[rectangle, draw] (specification) [above =1cm of use_cases] {Specification};
    \node[rectangle, draw, align=right] (design_language) [above =1cm of physical] {Design \\ Language};
    
    % Arrows for relationsships
    \path[->, line width=0.4mm]
        (user_stories) edge (understanding)
        (conceptual_scenarios) edge (understanding)
        (concrete_scenarios)   edge [bend right] (conceptual)
                               edge node [right] {prototyping} (envisionment)
        (use_cases)            edge (specification)
        (understanding)        edge node [below left] {Requirements/Problem} (conceptual)
        (conceptual)           edge node [below] {Conceptual model etc.} (physical)
        (envisionment)         edge (physical)
        (physical)             edge (design_language)
    ;
  \end{tikzpicture}
  \caption[Types of scenarios in relation to the Y-model]{The four different types of scenarios in relation to the Y-model. Note, that the \emph{evaluation} phase is not shown to make the figure more readable. \cite[p. 67]{benyon_14}}
  \label{fig:scenarios}
\end{figure}


\begin{definition}[Scenario Corpus] \label{def:scenario_corpus} \index{Scenario Corpus}
  A set of user stories picked out of the many collected, which together portray a cohesive and exhaustive description of the problem domain. This will highlight the different \emph{dimensions} over which the system to be designed spans, such as the functions, content, aesthetics and more. \cite[p. 67-68]{benyon_14}
\end{definition}

While the PACT framework of section \ref{sec:pact} is used to critique the scenarios, the fast growing amount of scenarios needs to be managed. To help collaboration and conveying the point of the scenarios they can be annotated with a description of the personas included, the activities covered, and the key points of the scenario among other things. These descriptions can be further annotated with meta data, such as the author of and their rationale for the scenario, its changehistory, and the domains to which it generalizes. Furthermore how the scenarios are related to eachother, such as what scenario spawned another, can be very valuable \cite[p. 70,72]{benyon14}

\begin{definition}[Endnote] \label{def:endnote} \index{Endnote}
 \cite[p. 70]{benyon14}
\end{definition}
\begin{method}[Trade-offs and claims analysis] \label{meth:trade-offs_and_claims} \index{Trade-offs and claims analysis}
 \cite[p. 70]{benyon_14} 
\end{method}
\begin{method}[Object-Action analysis] \label{meth:object-action_analysis} \index{Object-Action analysis}
  For every sceneraio it is broken down into its different activities and the following table
  \begin{center}
    \begin{tabular}{c|c|c|c|c}
      \hline
      Activity & Consists of sub-activites & Action & Object & Comments
      \\ \hline
    \end{tabular}
  \end{center}
  where the \emph{sub-activities} are the small and atomic steps needed to complete the activity. The \emph{actions} are the functions used, buttons pressed and interactions done, while the \emph{object} are the actual object on which the action is applied to. The comments are thoughts and claims of problems in the current design based on the rest of the row.

  From this the following table is created with every activity noted and its occurences counted.
  \begin{center}
    \begin{tabular}{c|c}
      \hline
      All actions & All objects
      \\ \hline
    \end{tabular}
  \end{center}
  Based on this actions and objects can begin to be merged into one, if they have enough conceptually or functionally parallels. This can be used as the basis to manifest the conceptual model of tool \ref{tool:conceptual_model}. \cite[p 198]{benyon14}
\end{method}

Together all of this can be used to derive the conceptual model and design language of section \ref{sec:conceptual_model}, and from method \ref{meth:object-action_analysis} can tool \ref{tool:conceptual_model} be derived. \cite[p. 67]{benyon14}

