\section{Understanding} \label{sec:understanding}
As part of the \emph{understanding} phase of the Y-model (framework \ref{fw:y_model}) the designer wants to gain knowledge and insight into the problem domain for which is designed.Gaining such an understanding is twofold: gathering information and making the information easy to manage and understand.

\subsection{Data Gathering} \label{sec:data_gathering}
Many techniques have proven useful for gathering information on the different focuspoints of the designer.

\begin{method}[Interview] \label{meth:interview} \index{Interview}
  Directly talking and asking questions to stakeholders is one key way to learn
  of their needs, requirements and problems. 
  
  Interviews can be done at several stages of the process: during the data
  analysis or getting evaluation of a the data analysis, envisionment or on a
  prototype. Interviews can take on many different forms, which all have different pros and
  cons.
  \cite[p. 142-146]{benyon_14}
\end{method}

\subparagraph{Structured Interview / Survey} \index{Structured Interview}
A fully prestructured interview, where the interviewee only answers premade
questions. While unable to follow up on unexpected answers, they are easy to
carry out, which creates a high-quantity of responses, which can be analysed
statistically.
\cite[p. 142]{benyon_14}

\subparagraph{Semistructured Interview} \index{Semistructured Interview}
An interview with prepared questions and topics to be asked and covered, while
the interviewer still allows for the discussion to digress in relevant
directions. Preperations mainly consist of creating a list of topics and prompts
to cover, and while more complicated for the interviewer, the data is more
nuanced and of higher-quality.
\cite[p. 143]{benyon_14}

\subparagraph{Unstructured Interview} \index{Unstructured Interview}
The interviewer makes no preperation going into the interview, either because
little information of the subjectmatter can be researched, or to minimize any
biases and keep the interviewer open to the interviewee's answers.
\cite[p. 143]{benyon_14}

\subparagraph{Focus Group Interview} \index{Focus Group Interview}
\todo
\begin{method}[Observation] \label{meth:observation} \index{Observation}
  The designer only observes the user's behaviour without physically or verbally interacting with the subject. This works especially well to uncover the user's tacit knowledge (definition \ref{def:tacit_knowledge}). In the the Y-model, observations can be used in the field, observing subjects everyday behaviour, as part of the \emph{understanding}-phase; and they can be used in controlled environments, observing the subjects attempting to solve a predefined task, as part of the \emph{evaluation}-phase. \cite[p.247-248]{rogers}

  The following are a set of bulletpoints for the designer to answer the questions they might have \cite[p.249]{rogers}
  \begin{itemize}
    \item \emph{Space}: The layout and contents of the environment.
    \item \emph{Actors}: Relevant details of the people involved.
    \item \emph{Goals}: The goals of the actors.
    \item \emph{Activities}: What people are doing.
    \item \emph{Artefacts}: Artefacts used throughout (model \ref{mod:artefact}).
    \item \emph{Time}: The order of the activities and events.
    \item \emph{Context}: The context the activities observed take place within.
    \item \emph{Feelings}: The mood of the actors individually and as a group.
  \end{itemize}
\end{method}
\begin{method}[Contextual Interview] \label{meth:contextual_interview} \index{Contextual Inteview}
  In its most basic form \emph{Contextual Interview} is the combination of interviewing the users (method \ref{meth:interview}) while also observing them (method \ref{meth:observation}) in the context of their work. When the analyst asks the subject their questions is highly dependant on the situation the contextual interview is conducted within, but the power of the contextual interview lies in the analyst being able to ask the subject about their actions and get the subject's feedback on the analyst's interpretations. \cite[p. 273-276]{benyon_10}

  A typical contextual interview could look as follows
  \begin{itemize}
    \item 15 minutes: Introductions, permissions, and explaining the contextual interview's focus
    \item 2 to 3 hours: Observations intermixed with conversation and questions
    \item 15 minutes: The analyst reviews their reflections with the subject
  \end{itemize}
  though if the activities observed would be nonrepresentative if the analyst intervened with questions, then it makes more sense to ask the subject about their activities and thoughtprocess after finishing the activity or the whole observation. \cite[p. 275-276]{benyon_10}  
\end{method}

\begin{method}[Artefact Collection] \label{meth:artefact_collection} \index{Artefact!Collection}
  Collecting artifacts either physically or copied by making a sketch or taking picture of them, if you are unable to retreive the artifact from the site. It is also very important for the designer to ask the users to get a common understanding of the artefact before taking it away from its context and annotating it. \cite[p. 286]{benyon_10}
\end{method}

\subsection{Data Analysis} \label{sec:data_analysis}
Sitting with raw and a massive amount of data gathered, the information is too unwieldy for later use. To sift through the information and condense it the methods below have proven useful. After having done the analysis, criteria for the system to be designed will be clear to the designer. Furthermore together with the methods below as part of the analysis personas user stories and personas (tool \ref{tool:user_story}, \ref{tool:persona}) are created, which are invaluable tools in the later design process.

\begin{method}[Affinity Diagram] \label{meth:affinity_diagram} \index{Affinity Diagram}
  
\end{method}

Together with analysing artefacts (model \ref{mod:artefact}) the following four models below are part of \emph{Work modelling}, where the different aspects of the users environment are captured \cite[p. 277]{benyon_10} \index{Work modelling}

\begin{model}[Flow Model] \label{mod:flow_model} \index{Flow Model}
  
\end{model}

\begin{method}[Sequence Model] \label{meth:sequence_model} \index{Sequence Model}
  
\end{method}

\begin{model}[Cultural Model] \label{mod:cultural_model} \index{Cultural Model}
  
\end{model}

\begin{model}[Physical Model] \label{mod:physical_model} \index{Physical Model}
  
\end{model}

