\section{Conceptual Model} \label{sec:conceptual_model}


\begin{concept}[Conceptual Model] \label{conc:conceptual_model} \index{Conceptual!Model}
  
\end{concept}

\begin{tool}[Conceptual Model] \label{tool:conceptual_model} \index{Conceptual!Model}
  One way to manifest the conceptual model of concept \ref{conc:conceptual_model} is by representing it as an entity-relationsship model. \cite[p. 199-200]{benyon_14}
\end{tool}

\begin{definition}[Metaphor] \label{def:metaphor} \index{Methaphor} \index{Interface!Metaphor}
 An analogy to a physical object, concepts environment or activities from a source domain to the target domain of the system. These metaphors carry with them preconceptions about what objects are and what can be done with then. \cite[p. 40]{rogers} \cite[p. 191-192]{benyon14} 
\end{definition}


While metaphors almost always are needed to convey a new domain to a user, they are also a two-edged sword. Both the source and the target domain have concepts and features, and the goal using a metaphor is to help users correctly expect the concepts and features of the target domain. The relationsship between the source and the target domain must hence be analysed thouroughly, since otherwise users may get confused. This can happen both with too many or inappriopriate connotations in a metaphor, which results in \emph{conceptual baggage}, i.e. suggested functionality, that is not part of the target domain, or it can happen when the metaphor contains too few features mapped to the target. \cite[p. 192, 195-196]{benyon14} \cite[p. 181-182]{anderson} \index{Conceptual!Baggage}

\begin{remark}[Blend] \label{rem:blend} \index{Blend}
  In most practical applications, when a metaphor is used, the final design draws from both the source and the target domain, where some features of the target domain have no equivalent in the source while others are left out of the design. These are hence \emph{blends}, since they draw from at least two domains. \cite[p. 194]{benyon14}
\end{remark}

The following are a few principles for designing metaphors. \cite[p. 196]{benyon14}
\begin{itemize}
  \item \emph{Coherence}:
    Metaphors should not be mixed and the consistency of the blend's structure should be maintained.
  \item \emph{Transparency}:
    The domain from which the different parts of the blend originate and why they work should be apparent to the user
  \item \emph{Stricly needed}:
    The blend should only contain what is strictly necessary for it to work.
  \item \emph{Topology}:
    How the concepts are organized and structured in both the source and the target should be the same - i.e. they have similar topologies.
  \item \emph{Representation}:
    Visual representations of objects and actions do not need to be realistic, but can be metaphorical.    
\end{itemize}

\begin{definition}[Design Language] \label{def:design_language} \index{Design language}
  As part of \emph{physical} design a design language is a cohesive assembly of  \todo  \cite[p. 203-204]{benyon_14}
\end{definition}


\todo

\begin{concept}[Mental Model] \label{conc:mental_model} \index{Mental!Model}
  As users read about, observe and interact with a system they develop a \emph{mental model} of how the system operates and the relationsship between its parts. Some key properties of mental models are that they are
  \begin{itemize}
    \item incomplete and finer in detail on some parts compared to others.
    \item unstable, as people easily forget details.
    \item not bounded to the specific system, as users can confuse it with the mental model of similar systems or reapply the current on other systems.
    \item unscientific and sometimes with \emph{superstitious} elements.
    \item parsimonious, i.e. users gladly do more physical actions to minimize mental efforts.
    \item possibly based on inappropriate analogies.
  \end{itemize}
  Furthermore users can in a \emph{mental simulation} run their mental model, predicting with limited accuracy the outcome of a set of actions. \cite[p. 31-32]{benyon14} \cite[p. 86-88]{rogers} \index{Mental!Simulation}

  If a user does not have a good mental model of a system, then most interactions performed by them will be memorized and they will be unable to recover from failure and errors - not understanding what went wrong. \cite[p. 31-32]{benyon14}
\end{concept}

In figure \ref{fig:mental_model} the relationsship between the designer, the user, and the system is shown. What the system does may or may not be aligned with the mental model of the designer, which is completely different as to how the users may think the system behaves. \cite[p. 31]{benyon14} 

\begin{figure}[ht!]
  \centering
  \begin{tikzpicture}
    % Mental Model
    \node[cloud, draw, aspect=2, align=left] (mental_model) {\bf Mental Model \\};
    \node[rectangle, fill=gray, draw, aspect=2] (system) [below=-1.2cm of mental_model] {The system};
    
    % Designers and users
    \node[rectangle, fill=GreenYellow, draw, aspect=2, minimum width=2.2cm]
          (designers) [left=1.5cm of mental_model] {\bf Designers};
    \node[rectangle, fill=GreenYellow, draw, aspect=2, minimum width=2.2cm]
          (users) [right=1.5cm of mental_model] {\bf Users};

    % What actually happens
    \node[rectangle] (what_happens) [below=1cm of mental_model] {What actually happens};
          
    \path[->, line width=0.4mm]
        (designers) edge [bend left=30] node [above] {conception} (mental_model)
        (mental_model) edge [bend right=30] node [below] {conception} (users)
        (users) edge [bend right=30] node [above] {interaction} (mental_model)
        (system) edge (what_happens)
        ;
  \end{tikzpicture}
  \caption[Conceptual Model]{The conceptual model \cite[p. 31]{benyon14}}
  \label{fig:mental_model}
\end{figure}


The goal of the designer is to design the system, such that the users will easily develop a mental model close to the conceptual model. This can be achieved by making the system more transparent with useful feedback, guidance, and good choice in how the users interact. Furthermore designers should develop a design language, use metaphors that communicate the conceptual model well, and find a clear, logical, and consistent conceptual design (section \ref{sec:conceptual_design}) then the users' will easier align their mental model with the intended conceptual model \cite[p. 86-88]{rogers} \cite[p. 32]{benyon14}
