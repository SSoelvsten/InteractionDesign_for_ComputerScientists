\section{Y-model} \label{sec:y_model} \index{Y-Model}
The Y-model is a characterization of the overall design process, dividing the designer's work into the activities: \emph{Evaluation}, \emph{Understanding}, \emph{Design}, and \emph{Envisionment}. As visualized in figure \ref{fig:y_model} the design process is non-linear and evaluation is central to the process. The designer may start in any of the four activities and everything by the designer is evaluated before moving onto the next activity. \cite[p. 49]{benyon14}

\begin{figure}[ht!]
  \centering
  \begin{tikzpicture}
    \node[cloud, draw, aspect=2] (evaluation) {Evaluation};
    \node[cloud, draw, aspect=2] (envisionment) [above left=1cm of evaluation] {Envisionment};
    \node[cloud, draw, aspect=2] (understanding) [above right=1cm of evaluation] {Understanding};
    \node[cloud, draw, aspect=2, align=left] (design) [below =1cm of evaluation] {Design \\ \\ \\};
    \node[cloud, draw, fill=white, aspect=2, align=right] (conceptual) [below left =-1.7cm of design] {Conceptual \\ Design};
    \node[cloud, draw, fill=white, aspect=2, align=left] (physical) [below right =-1.7cm of design] {Physical \\ Design};
    \path[<->, line width=0.6mm]
        (evaluation) edge (envisionment)
                     edge (understanding)
                     edge (design)
    ;
  \end{tikzpicture}
  \caption{The Y-model visualized \cite[p. 49]{benyon_14}}
  \label{fig:y_model}
\end{figure}


\subsection{Evaluation} \index{Evaluation}
Everything produced by the designer, be it understanding of the users needs and context, abstract design ideas or concrete prototypes, has to be evaluated with the end users. The type of evaluation varies heavily based on what is evaluated and the focus of the designer. \cite[p. 53-54]{benyon14}

\subsection{Understanding} \index{Understanding}
Research of stakeholders (definition \ref{def:stakeholder}) and their activities and the contexts within the problem domain. On an abstract level the designer wants to identify the stakeholders goals, requirements, wishes and needs, together with the restrictions on technologies and people due to the surrounding context. \cite[p. 50-51]{benyon14}

\subsection{Design}
Any part of the design, which are not immediately resulting in a physical product, but rather on a more abstract level specifies the product. \cite[p. 50]{benyon14}

\paragraph{Conceptual Design} \index{Conceptual Design}
On an abtract level the overall purpose of the system. This includes what functionality, structure, and information is provided by the system designed. These are specified together with finding a conceptual model (concept \ref{conc:conceptual_model}) to clearly communicate the design to the user. This can be manifested in a specification and vision of the system or through other ways, but should be as independent of implementations as possible. \cite[p. 51, 188, 206]{benyon14}
    
\paragraph{Physical Design} \index{Physical Design}
Going from the abstract to the concrete, the designer specifies the sequence of interactions, the presentation of information and functions, and the physical feel of the system. It consists of the following three components
\begin{itemize}
\item {\bf Operational Design} \index{Operational Design}
  The way functions and information is structured relative to each other throughout the system
\item {\bf Representational Design} \index{Representational Design}
  The overall feel and look of the design
\item {\bf Interaction Design} \index{Interaction Design}
  The structure and sequence of the interactions between humans and the system. \cite[p. 51-53, 188, 206]{benyon14}
\end{itemize}
      
\subsection{Envisionment} \index{Envisionment}
Making the design ideas real and bringing them physically into the world, the designers create everything from simple sketches to full blown high-fidelity prototypes (definition \ref{def:high-fidelity_prototype}). With these in hand the designers can concretely evaluate their ideas with eachother and with the users. \cite[p. 53-54, 166]{benyon14}
