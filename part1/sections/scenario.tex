\section{Scenario based design} \label{sec:scenario_based_design}
Useful in all four stages of the Y-model (section \ref{sec:y_model}), scenarios and personas are one of the most fundamental and popular techniques to designing interactive systems. \cite[p. 62]{benyon14}

\begin{tool}[User Story (Scenario)] \label{tool:user_story} \index{User Story (Scenario)}
  A real-world rendition of the activities, experiences, knowledge etc. of a subject. It is of very high detail, and it can be recorded in many different formats, such as video, text, interview and much more. This is useful to gain an \emph{understanding} of the stakeholders needs and more. \cite[p. 62-63]{benyon_14}
\end{tool}

\begin{tool}[Conceptual Scenario] \label{tool:conceptual_scenario} \index{Conceptual Scenario}
  An abstract description of the users and the activities that the system supports. This high form of abstraction is great for imagining solutions in the \emph{Conceptual Design} phase and to condense knowledge in the \emph{understanding} phase (framework \ref{fw:y_model}). Furthermore the level of abstraction defines a domain of reasonable designsolutions without rejecting any reasonable designideas early, which is of great value later in the process. \cite[p. 63]{benyon_14}
\end{tool}

\begin{tool}[Concrete Scenario] \label{tool:concrete_scenario} \index{Concrete Scenario}
  A concretization of how a designsolution would work within a specific context and situation as part of the \emph{Conceptual Design}. These to a greater or lesser extend further define interface designs and the different types of and relationship between functions in the system. \cite[p. 64]{benyon_14}
\end{tool}
\begin{method}[Use Cases (Scenario)] \label{meth:use_cases} \index{Use Cases (Scenario)}
  
\end{method}


A central tool used in the scenario based design is also the \emph{persona}, which can especially enhance both concrete scenarios and use cases and in parts also conceptual scenarios and for revising user stories

\begin{tool}[Persona] \label{tool:persona} \index{Persona}
  A fully fleshed out and concrete characterization of a type of user designed for, including most importantly the persona's background, prerequisites, and goals, but also especially their human characteristics. The more detailed these are the easier they make it for the designer to emphasize with them and as a result design for the actual user. \cite[p. 55]{benyon_14}
\end{tool}

With all four types of scenarios we see all activities of the Y-model are covered. Many user stories show the users needs of the system, abstracted into very few conceptual scenarios defining the vision for the solution. These few then create many concrete scenarios generating concrete solutions, from them many more use cases specifying the exact details of the functions of the system. This relation between the stories, the design process and the Y-model is visualized in figure \ref{fig:scenarios}. \cite[p. 66, 196]{benyon14}

\begin{figure}
  \centering
  \begin{tikzpicture}
    % Scenarios
    \node[rectangle, fill=GreenYellow, draw, aspect=2] (user_stories) {User Stories \vphantom{p}};
    \node[rectangle, fill=GreenYellow, draw, aspect=2, align=right] (conceptual_scenarios) [right=1.7cm of user_stories] {Conceptual \\ Scenario};
    \node[rectangle, fill=GreenYellow, draw, aspect=2, align=left] (concrete_scenarios) [right=1.7cm of conceptual_scenarios] {Concrete \\ Scenario};
    \node[rectangle, fill=GreenYellow, draw, aspect=2] (use_cases) [right=1.7cm of concrete_scenarios] {Use Cases \vphantom{p}};
    
    % Lines between scenarios
    \path[-, line width=0.3mm]
        (user_stories) edge node [above] {$* \quad \quad 1$} (conceptual_scenarios)
        (conceptual_scenarios) edge node [above]  {$1 \quad \quad *$} (concrete_scenarios)
        (concrete_scenarios) edge node [above]  {$* \quad \quad *$} (use_cases)
    ;
    
    % Y-model
    \node[cloud, draw, aspect=2] (envisionment) [below =1.2cm of concrete_scenarios] {Envisionment};
    \node[cloud, draw, aspect=2] (understanding) [left =2.1cm of envisionment] {Understanding};
    \node[cloud, draw, fill=white, aspect=2, align=left] (physical) [below  =4.8cm of use_cases] {Physical \\ Design};
    \node[cloud, draw, fill=white, aspect=2, align=right] (conceptual) [left =3.7cm of physical] {Conceptual \\ Design $\quad$};
    
    % Text nodes
    \node[rectangle, draw] (specification) [above =1cm of use_cases] {Specification};
    \node[rectangle, draw, align=right] (design_language) [above =1cm of physical] {Design \\ Language};
    
    % Arrows for relationsships
    \path[->, line width=0.4mm]
        (user_stories) edge (understanding)
        (conceptual_scenarios) edge (understanding)
        (concrete_scenarios)   edge [bend right] (conceptual)
                               edge node [right] {prototyping} (envisionment)
        (use_cases)            edge (specification)
        (understanding)        edge node [below left] {Requirements/Problem} (conceptual)
        (conceptual)           edge node [below] {Conceptual model etc.} (physical)
        (envisionment)         edge (physical)
        (physical)             edge (design_language)
    ;
  \end{tikzpicture}
  \caption[Types of scenarios in relation to the Y-model]{The four different types of scenarios in relation to the Y-model. Note, that the \emph{evaluation} phase is not shown to make the figure more readable. \cite[p. 67]{benyon_14}}
  \label{fig:scenarios}
\end{figure}


\begin{definition}[Scenario Corpus] \label{def:scenario_corpus} \index{Scenario Corpus}
  A set of user stories picked out of the many collected, which together portray a cohesive and exhaustive description of the problem domain. This will highlight the different \emph{dimensions} over which the system to be designed spans, such as the functions, content, aesthetics and more. \cite[p. 67-68]{benyon_14}
\end{definition}

While the PACT framework of section \ref{sec:pact} is used to critique the scenarios, the fast growing amount of scenarios needs to be managed. To help collaboration and conveying the point of the scenarios they can be annotated with the \emph{P}, \emph{A}, \emph{C}, and \emph{T} of the the \emph{PACT}, and the key points of the scenario among other things. These descriptions can be further annotated with meta data, such as the author of and their rationale for the scenario, its changehistory, and the domains to which it generalizes to further help organize collaboration. Furthermore how the scenarios are related to eachother, such as what scenario spawned another, can be very valuable \cite[p. 70,72]{benyon14}

\begin{definition}[Endnote] \label{def:endnote} \index{Endnote}
 \cite[p. 70]{benyon14}
\end{definition}
\begin{method}[Trade-offs and claims analysis] \label{meth:trade-offs_and_claims} \index{Trade-offs and claims analysis}
 \cite[p. 70]{benyon_14} 
\end{method}
\begin{method}[QOC]
  The \emph{QOC} works similarly to method \ref{meth:object-action_analysis}. Instead of the trade-offs the designer lists the design \emph{q}uestions raised, the design \emph{o}ptions that can and cannot answer these, and the \emph{c}riteria establised by the scenario. \cite[p. 69]{benyon14}
\end{method}
\begin{method}[Object-Action analysis] \label{meth:object-action_analysis} \index{Object-Action analysis}
  For every sceneraio it is broken down into its different activities and the following table
  \begin{center}
    \begin{tabular}{c|c|c|c|c}
      \hline
      Activity & Consists of sub-activites & Action & Object & Comments
      \\ \hline
    \end{tabular}
  \end{center}
  where the \emph{sub-activities} are the small and atomic steps needed to complete the activity. The \emph{actions} are the functions used, buttons pressed and interactions done, while the \emph{object} are the actual object on which the action is applied to. The comments are thoughts and claims of problems in the current design based on the rest of the row.

  From this the following table is created with every activity noted and its occurences counted.
  \begin{center}
    \begin{tabular}{c|c}
      \hline
      All actions & All objects
      \\ \hline
    \end{tabular}
  \end{center}
  Based on this actions and objects can begin to be merged into one, if they have enough conceptually or functionally parallels. This can be used as the basis to manifest the conceptual model of tool \ref{tool:conceptual_model}. \cite[p 198]{benyon14}
\end{method}

Together all of this can be used to derive the conceptual model and design language of section \ref{sec:conceptual_model}, and from method \ref{meth:object-action_analysis} can tool \ref{tool:conceptual_model} be derived. \cite[p. 67]{benyon14}
