\section{Agile Development} \label{sec:agile_development}


\begin{concept}[Agile Method] \label{conc:agile} \index{Agile Method}
  
\end{concept}

\begin{definition}[User Story] \label{def:user_story} \index{User Story}
  
\end{definition}

\begin{definition}[Product Owner] \label{def:product_owner} \index{Product Owner}

\end{definition}

\begin{definition}[Product Backlog] \label{def:product_backlog} \index{Product Backlog}

\end{definition}

\subsection{Phase 0 - Design comes before agile} \label{sec:design_before_agile}
The design process described and endorsed throughout this book in chapter \ref{chap:process} and \ref{chap:methods} are based on letting the designer freely explore the problem domain and designspace. Since any type of agile development or any other type of software development is based on commiting to short or long term goals, these workflows do not work for the design phase. In the particular case of Scrum described below, the user stories of definition \ref{def:user_story} cannot be created and hence no ability to commit to anything for a sprint. Because of this it is normal to have a \emph{Sprint 0} with no set goal and commitments, such that software development first starts, when designers have explored the possibilities and gained a focus. \cite[p. 27]{beyer}

By use of the user stories it is also possible to do the work of the designer in parallel to the development of the solution, but where the designers then are always one sprint ahead of the developers. \cite[p. 13]{beyer}

\subsection{Scrum} \label{sec:scrum} \index{Scrum}
Scrum is one of the most popular implementations of the agile development method.

\begin{definition}[Scrum Master] \label{def:scrum_master} \index{Scrum!Master}

\end{definition}

\begin{definition}[Sprint] \label{def:sprint} \index{Sprint}

\end{definition}

\begin{definition}[Sprint Backlog] \label{def:sprint_backlog} \index{Sprint!Backlog}

\end{definition}

\begin{definition}[Daily Scrum] \label{def:daily_scrum} \index{Daily Scrum}

\end{definition}

\begin{definition}[Sprint Review and Retrospective]  \label{def:sprint_review} \index{Sprint!Review and Retrospective}

\end{definition}
