\chapter{Introduction} \label{chap:introduction}
Computer Scientists at Aarhus University tend to have a \emph{love-hate} relationship with all their HCI courses - with the only exception, that there is no \emph{love} at all. It is extremely saddening that so many students dismiss ever delving into this fascinating field. Many especially are frustratd due to the lack of a text book, which is clear and cuts to the point. All the books and other material, that have been used in the courses and can be found in the references, are drowning the actual points and what is important in huge blobs of useless text. The following paragraph from a 400 page long textbook is a prime example of this offense

\blockcquote[p. 435]{rogers}{\it What to evaluate ranges from low-tech prototypes to complete systems; a particular screen function to the whole workflow; and from aesthetic design to safety features. For example, developers of a new web browser may want to know if users find items faster with their product, whereas developers of an ambient display may be interested in whether it changes people's behaviour. Government authorities may ask if a computerized system for controlling traffic lights results in fewer accidents or if a website complies with the standards required for users with disabilities. Makers of a toy ask if 6-year-olds can manipulate the controls and whether they are engaged by its furry case, and whether the toy is safe for them to play with. A company that develops personal, digital music players may want to know if the size, color, and shape of the casings are liked by people from different age groups living in different countries. A software company may want to asses market reaction to its new homepage design.}

\noindent While the first sentence is useful and the next one arguably too, the next 4 are completely useless. If the reader had not already gotten the point by the first example or second, they would certainly not have learned it by the time they reached the end of this monstrosity of a paragraph. Similarly in the 27 pages long paper by Lim and Tenenberg \cite{lim} the same style of writing is evident, where the reader can scan the text and can already reach a conclusion two pages prior to the author. Simultaneously several methods and concepts are only very vaguely defined or not at all. For example a \emph{rich picture} is in the text book of Benyon only defined with a mere mention to its existence and a reference to a figure with two examples. \cite[p. 51-52]{benyon_14} Furthermore the textbooks contradict eachother and themselves, such as when Benyon claims the distinction between \emph{conceptual} and \emph{physical} design is not clearly defined, he five lines later states that the disinction between the two is very important. \cite[p. 51]{benyon_14}

After a paragraph similar to the one quoted for being awful, this is the last time this book will ever indulge in being superflous and unreadable. This book is an attempt to cut to the point by clearly define all concepts, frameworks, and methods, compare them, and highlight their pros and cons. We hope this can make the experience of HCI for some students less frustrating, maybe even turn it from an excruviating experience to a delightful one. For the same reasons this book is divided into four parts: theory, examples, cases, and exercises. This way we wish that the theory will not be drowned in examples. With the exercises we want to provide small and focused samples of data and problems, where the theory can be applied and that the material is learned effectively. With the theory seperated into its own part, we see this book first and foremost as a small encyclopedia of the world of HCI, yet we hope with the examples and exercises, that we can produce a text books, which can be used in a course of interaction design in its own right.

Every key word and concept is attempted to be categorized as one of the following
\begin{itemize}
   \item \textbf{Definition}: Definition of basic underlying ideas and vocabulary.

   \item \textbf{Concept}: An abstract idea and/or generalization of repeated human behaviour.

% TODO: Should be removed fully, since frameworks are so big, they usually get sections?
%   \item \textbf{Framework}: An abstract conceptual structure, around which a lot of methods are designed.

   \item \textbf{Method}: A process for designers to attain knowledge, understanding, and/or conclusions

   \item \textbf{Tool / Model}: An entity, which is by the designer created to directly inform the later design process. The distinction to be made from a \emph{method} is that a physical, visual or textbased entity is created to be directly used later, rather than just the conclusions.
\end{itemize}

\vspace{2em}
% https://tex.stackexchange.com/questions/128636/center-hrule-in-the-middle-of-the-page
\noindent\hfil\rule{0.8\textwidth}{.4pt}\hfil

\vspace{2em}
\noindent This book has been created as a joined group effort of the students of Aarhus University. Thanks to everyone who has contributed with smaller or bigger sections, corrections, feedback, proof-reading and much more.

\begin{multicols}{2}
  Johannes Ernstsen

  Simon Friis Vindum
  
  \hfill
  \columnbreak
  
  Kira Kutscher
\end{multicols}

