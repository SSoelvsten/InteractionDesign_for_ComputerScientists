\subsubsection{Data Analysis}
\label{sec:1_data_analysis}
The main purpose of the data analysis step is to take the data gathered and extract conclusions about the system to be designed. After having done the analysis, criteria for the system to be designed will be clear to the designer. Furthermore in this step a lot of tools such as personas and user stories are created, which are valuable tools in the later design process.

Since always more knowledge is gained around the field for which you design, it is normal to turn back to this stage quite a lot to redo or add to prior work.

\begin{method}[Affinity Diagram] \label{meth:affinity_diagram} \index{Affinity Diagram}
  
\end{method}

\begin{method}[Rich Picture] \label{meth:rich_picture} \index{Rich Picture}
  
\end{method}

\begin{tool}[Persona] \label{tool:persona} \index{Persona}
  
\end{tool}

\begin{tool}[User Stories (Scenario)] \label{tool:user_stories} \index{User Stories (Scenario)}
  
\end{tool}

\begin{figure}
  \centering
  % TODO : Recreate the figure from Benyon
  \caption{The four different types of scenarios \cite[p. 67]{benyon_14}}
  \label{fig:scenarios}
\end{figure}

\begin{definition}[Flow Model] \label{meth:flow_model} \index{Flow Model}
  
\end{definition}

\begin{definition}[Sequence Model] \label{meth:sequence_model} \index{Sequence Model}
  
\end{definition}
